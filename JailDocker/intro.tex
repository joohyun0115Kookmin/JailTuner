\section{Introduction} \label{sec:introduction}
%$$$$$$$$$$$$$$$$$$$$$$$$$$$$$$$$$$$$$$$$$$$$$$$$$$$$$$$$$$$$$$$$$$$$$$$$$$$$$$$$
%$$$$$$$$$$$$$$$$$$$$$$$$$$$$$$$$$$$$$$$$$$$$$$$$$$$$$$$$$$$$$$$$$$$$$$$$$$$$$$$$
%Background : 스파크 -> cloud가 아닌 scale-up server 에서의 scalability에 대한 연구가 필요해짐
%$$$$$$$$$$$$$$$$$$$$$$$$$$$$$$$$$$$$$$$$$$$$$$$$$$$$$$$$$$$$$$$$$$$$$$$$$$$$$$$$
\ifkor
빅데이터 처리하는데 많이 사용되는 framework 중 하나는 스파크이다.
최근 효율적인 Spark를 이용하여, large scale data에서 원하는 
정보만 추출한 데이터를 science fields와 machine learning에서 활용되기 시작하였다.
Science fields와 machine learning에서 사용되는 환경은 보통 cloud로 구성된 환경이 아닌,
scale-up server 환경이 많이 이용된다[].
뿐만 아니라, 점차 core수와 메모리 크기가 증가함[]에 따라, 
Scale-up 환경도 중요해 지고 있으며, 
scale-up 서버에서의 스파크에 대한 scalalbility 역시 중요해지고 있다. 
\else

\fi

%$$$$$$$$$$$$$$$$$$$$$$$$$$$$$$$$$$$$$$$$$$$$$$$$$$$$$$$$$$$$$$$$$$$$$$$$$$$$$$$$
%$$$$$$$$$$$$$$$$$$$$$$$$$$$$$$$$$$$$$$$$$$$$$$$$$$$$$$$$$$$$$$$$$$$$$$$$$$$$$$$$
%Problem : scale-up server에서 시스템으로 구성된 scalability가 없음 
% 2가지 관련 연구가 있음.
% 1. 24코어 이하의 서버에서의 Scalability 분석을 하였으나 해결책을 제안하지 않았음.
% 2. HPC(100이상) 으로 분석하였으나 flie system 관점으로 분석하였음.:메인 병목지점은 파일 시스템
% 3. Scalable한 파일 시스템을 사용한 후 Scalability에 대한 분석한 결과와 해결방법은 없음.
%$$$$$$$$$$$$$$$$$$$$$$$$$$$$$$$$$$$$$$$$$$$$$$$$$$$$$$$$$$$$$$$$$$$$$$$$$$$$$$$$
\ifkor
이처럼 Scale-Up 서버를 위한 scalability 연구들이 진행되고 있다.
그 중 Single node로 된 scale-up 서버에서의 아파치 스파크의 scalalbility에 가장 많은 
영향을 미치는 요소는 GC(Garbage Collecter)와 memory access latency이다[].
A.J. Awan et al.은 스파크의 GC의 serialize 문제점을 parrale GC로 수정하여 성능차이를 
보였다.
하지만, 이러한 방법 모두 GC가 가지고 있는 근본적인 성능 저하 때문에 여전히 
scalability 문제가 있다(section 2). 
또한 Spark이외에도 NUMA 구조 때문에 발생하는 memory access latency 문제를 해결하기 위해,
여러 NUMA 밸런싱[][]에 대한 연구가 진행되고 있다. 
\else


\fi


%$$$$$$$$$$$$$$$$$$$$$$$$$$$$$$$$$$$$$$$$$$$$$$$$$$$$$$$$$$$$$$$$$$$$$$$$$$$$$$$$
%본 연구에서 분석한 결과와 제안하는 방법으로 향상된 성능
%$$$$$$$$$$$$$$$$$$$$$$$$$$$$$$$$$$$$$$$$$$$$$$$$$$$$$$$$$$$$$$$$$$$$$$$$$$$$$$$$
%$$$$$$$$$$$$$$$$$$$$$$$$$$$$$$$$$$$$$$$$$$$$$$$$$$$$$$$$$$$$$$$$$$$$$$$$$$$$$$$$
\ifkor
본 연구는 스파크 scalabiity에 가장 많은 영향을 미치는 GC와 memory access latency에 
대한 문제를 동시에 해결하기 위해, 파티션닝 기법을 제안한다.
이 방법은 코어와 메모리를 나누어 적은 코어와 적은 메모리로 실행시키도록 만들었다.
따라서, 적은 코어 그룹이 공유데이터를 대상으로 작업 하므로 근본적으로 GC에 의해 
thread들이 serialized 되는 현상을 줄일 수 있다. 
또한 파티션된어 NUMA 소켓의 로컬 메모리만 접근하므로 memory access latency 줄일 수 있다. 
뿐만아니라 이러한 방법은 Scale-up 서버의 shared-memory 시스템에서 사용되는
운영체제의 근본적인 문제인 공유 때문에 발생하는
문제들(e.g, lock contention[Bonsai, Radix], cache communication overhead[Oplog,FC], single address space problem)을
함께 제거할 수 있다. 
우리는 docker container를 이용하여 파티션닝 기법을 구현하였고, 그 결과 shared-memory 시스템을 마치
distributed-system 처럼 구성하여 scalability를 향상 시켰다.
\else


\fi



%$$$$$$$$$$$$$$$$$$$$$$$$$$$$$$$$$$$$$$$$$$$$$$$$$$$$$$$$$$$$$$$$$$$$$$$$$$$$$$$$
%본 연구에서 분석한 결과와 제안하는 방법으로 향상된 성능
%$$$$$$$$$$$$$$$$$$$$$$$$$$$$$$$$$$$$$$$$$$$$$$$$$$$$$$$$$$$$$$$$$$$$$$$$$$$$$$$$
%$$$$$$$$$$$$$$$$$$$$$$$$$$$$$$$$$$$$$$$$$$$$$$$$$$$$$$$$$$$$$$$$$$$$$$$$$$$$$$$$
\ifkor
우리의 Docker container 기반 방법은 Scale-Up server에서 최적의 성능을 내기 위한 방법이다. 
우리의 방법을 적용한 결과 
The evaluation of the proposed apporch on a 120 core system reveals that 
the execution times could be improved by 1.7x, 1.6x, and 2.2x for 
a big data workload- , respectively.
\else


\fi

%$$$$$$$$$$$$$$$$$$$$$$$$$$$$$$$$$$$$$$$$$$$$$$$$$$$$$$$$$$$$$$$$$$$$$$$$$$$$$$$$
%본 연구에서 기여한 것 : 
% 1. 100코어 이상의 scale-up 서버에서의 scalability 측정 및 분석
% 2. 도커 파티션 기법을 활용한 scalability 향상 방법 제안
%$$$$$$$$$$$$$$$$$$$$$$$$$$$$$$$$$$$$$$$$$$$$$$$$$$$$$$$$$$$$$$$$$$$$$$$$$$$$$$$$
\ifkor

\textbf{Contributions.} Our research makes the following contributions:
\begin{itemize}
\item 
우리는 120코어로 구성된 scale-up 서버에서의 Spark에 대한 scalability를 분석하였다. 
분석 결과, pararell GC 통해 약간의 성능을 향상시킬 수 있었으나, 결국 60코어 이상되는 부분 부터는 
더 이상 성능향상은 없었다. 
\item 
우리는 도커기반의 파티션닝 방법을 120코어 시스템에 적용하여 BigDataBench의 
5가 워크로드에 대한 scalability 문제를 해결하였다. 
Our design improved throughput and execution time from 1.5x through 2.7x on 120 core.
\end{itemize}

\else

\fi


%$$$$$$$$$$$$$$$$$$$$$$$$$$$$$$$$$$$$$$$$$$$$$$$$$$$$$$$$$$$$$$$$$$$$$$$$$$$$$$$$
%$$$$$$$$$$$$$$$$$$$$$$$$$$$$$$$$$$$$$$$$$$$$$$$$$$$$$$$$$$$$$$$$$$$$$$$$$$$$$$$$
%Mapping
%$$$$$$$$$$$$$$$$$$$$$$$$$$$$$$$$$$$$$$$$$$$$$$$$$$$$$$$$$$$$$$$$$$$$$$$$$$$$$$$$
The rest of this paper is organized as follows.
Section 2 describes the test-bed and spark scalability problem.
Section 3 describes the our partitioning approach and 
Section 4 shows the results of the experimental evaluation. 
Section 5 describes related works. 
Finally, section 6 concludes the paper.

