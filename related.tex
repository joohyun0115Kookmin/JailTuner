\section{Related work} \label{sec:RelatedWork}

%$$$$$$$$$$$$$$$$$$$$$$$$$$$$$$$$$$$$$$$$$$$$$$$$$$$$$$$$$$$$$$$$$$$$$$$$$$$$$$$$
%Paragraph 1:Linux Scalability의 연구에 대한 설명
%$$$$$$$$$$$$$$$$$$$$$$$$$$$$$$$$$$$$$$$$$$$$$$$$$$$$$$$$$$$$$$$$$$$$$$$$$$$$$$$$
\noindent 
\textbf{Apache Spark Scalability.}
To improve the Spark scalability, researchers have attempted to optimize for
scale-out server~\cite{Ousterhout2015MSP}~\cite{Maas2016THL} or to optimize scale-up server~\cite{Ahsan2016SVS}~\cite{Chaimov2016SSH}.
Our research belongs to optimizing for scale-up server to eliminate GC overheads
and to enhance the locality on NUMA architecture.
However, previous studies did not considered Docker container-based partitioning,
which can clearly reduce memory contention, and it can maximize locality of memory access.
Furthermore, it can easily combine other container management solutions.

%$$$$$$$$$$$$$$$$$$$$$$$$$$$$$$$$$$$$$$$$$$$$$$$$$$$$$$$$$$$$$$$$$$$$$$$$$$$$$$$$
%Paragraph 1:Manycore Scalability의 연구에 대한 설명
%$$$$$$$$$$$$$$$$$$$$$$$$$$$$$$$$$$$$$$$$$$$$$$$$$$$$$$$$$$$$$$$$$$$$$$$$$$$$$$$$
\noindent
\textbf{Scale-up Server Scalability.}
To improve the Spark scalability, researchers have attempted to apply distributed
system concepts to shared
memory systems~\cite{Baumann2009Barrelfish}~\cite{SilasBoydWickizerPth}.
Barrelfish~\cite{Baumann2009Barrelfish} creates a new operating system for efficient cache-coherent
shared memory system by building an OS using message-based architecture.
Our research also brings about distributed system concepts, but our approach applies
to user level Spark framework instead of OS because OS can achieves 
performance scalability by commuting interface~\cite{Clements2013SCR}.
